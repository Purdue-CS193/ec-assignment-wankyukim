\documentclass{article}
\usepackage[utf8]{inputenc}

\title{CS 193 EC – Assignment for CS 182}
% Replace with your name
\author{yourName }
% Replace with today's month
\date{July 2020}

\begin{document}

\maketitle

\section{Logical Equivalences}
\begin{center}
 Show that  $\neg $ ($p \to q$) $\equiv $ $p \land \neg q$ using a truth table
\end{center}

\begin{displaymath}  % start unumbered math environment
  %
  % Start a table in math mode.  The |c|c|c|c|c|c| string is a
  % format string that says there will be 6 colunms in the table.  The
  % c's indicate that the data in each column will be centered (use l
  % for left justified and r for right justified).  The vertical bar
  % means that lines will be drawn between columns.  The trailing
  % \hline causes a horizontal line to be drawn across the top of the
  % table.
  %
  \begin{array}{|c|c|c|c|c|c|}\hline
    %
    % Each row of the table consists of data separated by "&" symbols.
    % Each row must end with "\\" to cause a newline.  A trailing
    % \hline will cause a line to be drawn under the row.  A double
    % \hline is often used to separate the table header from the rest
    % of the table. 
    %
    % Replace each "-" with its corresponding truth value
    % \mathbf{} bolds the value in the table
    % Each resulting column should be in bold
    %
    
    p & q & \neg p  & p \to q & \neg (p \to q) & p \land \neg q \\\hline
    - & - & - & - & \mathbf{-} & \mathbf{-} \\\hline
    - & - & - & - & \mathbf{-} & \mathbf{-}  \\\hline
    - & - & - & - & \mathbf{-} & \mathbf{-}  \\\hline
    - & - & - & - & \mathbf{-} & \mathbf{-}  \\\hline

  \end{array}
\end{displaymath}

\section{Notes}
\begin{flushleft}
One of the things I (Cindy) really enjoyed about CS 182 is that the course isn't entirely technical based or programming specific. CS 182 teaches you how to think logically using deduction and mathematical induction. I hope you enjoy this course as much as I did, and as always, please reach out to your TAs and/or professors as early as possible. Best of luck to your second semester!
\end{flushleft}
\end{document}
